\documentclass[12pt]{article}

\usepackage{sbc-template}
\usepackage{array}
\usepackage{graphicx,url}
\usepackage[tight,footnotesize]{subfigure}

%encoding
%--------------------------------------
\usepackage[utf8]{inputenc}
\usepackage[T1]{fontenc}
%--------------------------------------
 
%Portuguese-specific commands
%--------------------------------------
\usepackage[portuguese]{babel}
%--------------------------------------


%\usepackage[brazil]{babel}   
%\usepackage[latin1]{inputenc}  
\usepackage[export]{adjustbox}
%fontes personalizadas
\usepackage{enumerate}
\usepackage{bm}
%bibliotecas matem\'{a}ticas
\usepackage{amsmath}
\usepackage{amsfonts}
\usepackage{amssymb}
\usepackage{float}
\usepackage{pst-func}
\usepackage{pst-math}

\usepackage{color, colortbl}
\definecolor{Gray}{gray}{0.95}
\usepackage{kpfonts}
\usepackage[T1]{fontenc}

\usepackage{graphicx,url}

\usepackage{listings}
\definecolor{light-gray}{gray}{0.9}
\lstset{extendedchars=\true,
		inputencoding=ansinew,
		showspaces=false,
		showstringspaces=false,
		showtabs=false,
		tabsize=2,
		backgroundcolor=\color{light-gray}}
     
\sloppy

\usepackage{hyperref}
\hypersetup{
    unicode=false,          
    pdftoolbar=true,        
    pdfmenubar=true,       
    pdffitwindow=false,    
    pdfstartview={FitH},    
    colorlinks=true,       
    linkcolor=red,          
    citecolor=blue,         
    urlcolor=blue,        
    breaklinks=true
}


     
\sloppy

\title{Análise de Medidas de Centralidade Aplicadas as Redes Ópticas de Telecomunicações}

\author{Silvana Trindade\inst{1}, Guilherme Bizzani\inst{1}, Rodrigo Levinsk\inst{1}, Watson V. C. Junior\inst{1}}


\address{Universidade Federal da Fronteira Sul (UFFS)
  \email{\{syletri,gaiolerro,rd.levinsk,watsonmaster\}@gmail.com}
}

\begin{document} 

\maketitle

\begin{abstract}
 
\end{abstract}
%contextualiza?o
%gap
%proposito
%metodologia
%resultados
%conclusão
\begin{resumo} 
 A partir de um conjunto de redes ópticas de telecomunicações foi realizado uma análise sobre a influência de varáveis, como por exemplo o número de nós e ligações.
 Este artigo propõe o estudo de cinco medidas de centralidade: a centralidade de grau, de intermediação, de eficiência e proximidade.
\end{resumo}


\section{Introdução}

\section{Medidas de Centralidade}

%apresentar as redes reais utilizadas

\subsection{Centralidade de Eficiência}
%conceituar medidas
\subsection{Centralidade de Grau}
%conceituar medidas
\subsection{Centralidade de Intermediação}
%conceituar medidas
\subsection{Centralidade de Proximidade}
%conceituar medidas

\section{Metodologia}
%como foi feito
%descrever etapas
\begin{table}[htp]
\caption{CONJUNTO DE REDES REAIS DE REFERÊCIA}\label{tab:tab1}
\centering
\begin{tabular}{l*{9}{l}r}
\hline\rowcolor{Gray}
Número & Rede & \textit{N} & \textit{L} & $\langle \delta \rangle$ & $C_B$ & $C_R$ & $C_E$ & $C_P$ & $C_G$\\ 
\hline
1 & Coronet & 75 & 97 & 2.59& 1034.95& 0.33& 0.07 & 0.11\\ 
2 &  Abilinecore &  10 &  13& 2.60& 12.58 & 0.30 & 0.33\\
3 &  BREN &  10 &  11& 2.20& 11.5 & 0.33 & 0.33\\
4 &  Learn &  10 &  11& 2.20& 11.5& 0.33 & 0.33\\
5 &  RNP &  10 &  12& 2.40& 10.50&  0.33 & 0.33\\
6 &  Metrona &  33 &  41& 2.48& 239.50&  0.16 & 0.17\\
7 &  ARNES &  17 &  20& 2.35& 74.83&  0.31 & 0.33\\
8 &  Germany &  17 &  26& 3.06& 47.93& 0.37 & 0.33\\
9 &  ARPANET &  20 &  32& 3.20& 35.40&  0.21 & 0.33\\
10 &  Sweden &  20 &  24& 2.40& 53.00& 0.21 & 0.20\\
11 &  LONI &  33 &  37& 2.24& 247.67&  0.19 & 0.11\\
12 &  INTERNET2 &  56&  61& 2.18& 631.42&  0.05 & 0.11\\
13 &  USA100 &  100 &  171& 3.42& 1720.56&  0.06 & 0.11
\end{tabular}
\hline
\end{table}
\section{Resultados e Discussões}

%exemplo de inserir img
% \begin{figure}[htp]
% \centering
% \includegraphics[width=6in,natwidth=516,natheight=65]{grafico.png}
% \caption{comentarios}
% \label{fig:fig1}
%\end{figure}

%graficos
%analises
\section{Conclusão e Trabalhos Futuros}

\bibliographystyle{sbc}
\bibliography{sbc-template}

\end{document}
